% Some helpful commands for making CS223A handouts
% Note: this file may not be exactly the same as
% versions from previous years.

% The document class options
\documentclass[11pt,oneside,notitlepage]{article}

% Packages
\usepackage{epsfig}
\usepackage{enumerate}
\usepackage{amsmath}
\usepackage{color}
\usepackage{ulem}
\usepackage{enumitem}

% Preamble (style parameters, etc)
\pagestyle{empty}

\setlength{\paperwidth}{8.5in}
\setlength{\paperheight}{11in}
\setlength{\textwidth}{6.5in}
\setlength{\textheight}{9.5in}

\setlength{\topmargin}{0in}
\setlength{\evensidemargin}{0in}
\setlength{\oddsidemargin}{0in}

\setlength{\headheight}{0in}
\setlength{\headsep}{0in}

% Command to display the header
\newcommand{\HandoutHeader}[1]{
\noindent{\Large{\textbf{Introduction to Robotics (CS223A)}} \hfill Handout {#1}} \rule[1.0mm]{6.5in}{1mm} {\Large (Winter 2016/2017)}
\newline\newline }

\newcommand{\HomeworkHeader}[2]{
\noindent{\newline\newline\newline\Large{\textbf{Introduction to Robotics (CS223A)}} \hfill Homework \#{#1}} \rule[1.0mm]{6.5in}{1mm}
\\
{\Large (Winter 2017/2018)} \hfill {\large Due: \textbf{#2}}\newline\newline }

\newcommand{\HomeworkSolutionsHeader}[2]{
\noindent{\newline\newline\newline\Large{\textbf{Introduction to Robotics (CS223A)}} \hfill Homework \#{#1} Solution} \rule[1.0mm]{6.5in}{1mm} {\Large (Winter 2017/2018)}\newline\newline }


%\newcommand{\HomeworkSolutionsHeader}[1]{
%{\Large{\bfseries Homework \#{#1} solutions}}
%}

\newcommand{\MidtermHeader}[1]{
\noindent{\Large{\bfseries Introduction to Robotics (CS223A)}}\newline
\rule[1.0mm]{6.5in}{1mm}
{\Large (Winter 2017/2018)} \newline\newline\newline
{\Large{\bfseries Midterm}} \hfill {\large Date: \textbf{#1}}\newline\newline\newline
}

\newcommand{\MidtermSolutionsHeader}[1]{
\newline
{\Large{\bfseries Midterm Solutions}} \hfill {\large Date: \textbf{#1}}\newline\newline\newline
}

\newcommand{\FinalHeader}[1]{
\noindent{\Large{\bfseries Introduction to Robotics (CS223A)}}\newline
\rule[1.0mm]{6.5in}{1mm}
{\Large (Winter 2017/2018)} \newline\newline\newline
{\Large{\bfseries Final}} \hfill {\large Date: \textbf{#1}}\newline\newline\newline
}

\newcommand{\FinalSolutionsHeader}[1]{
\newline
{\Large{\bfseries Final Solutions}} \hfill {\large Date: \textbf{#1}}\newline\newline\newline
}

% Helper commands

\newcommand{\q}{\ensuremath{\V{}{q}}}
\newcommand{\qdot}{\ensuremath{\V{}{\dot{q}}}}
\newcommand{\qddot}{\ensuremath{\V{}{\ddot{q}}}}
%\newcommand{\p}[2]{\ensuremath{{^{#1}{\bf P}}
\newcommand{\F}[1]{\ensuremath{\{#1\}}}
\newcommand{\R}[2]{\ensuremath{^{#1}_{#2}R}}
\newcommand{\T}[2]{\ensuremath{^{#1}_{#2}T}}
\newcommand{\U}[2]{\ensuremath{\hat{#1}_{#2}}}
\newcommand{\V}[2]{\ensuremath{^{#1}{\bf #2}}}
\newcommand{\degs}{\ensuremath{^\circ}}
\newcommand{\inv}{\ensuremath{^{-1}}}
\renewenvironment{matrix}{\left[\begin{array}}{\end{array}\right]}
\newenvironment{Tmatrix}{\begin{matrix}{rrrr}}{\end{matrix}}
\newcommand{\Col}[1]{\ensuremath{\begin{matrix}{r} #1 \end{matrix}}}
\newcommand{\Row}[1]{\ensuremath{\begin{matrix}{rrr}#1 \end{matrix}}}
\newcommand{\mat}[1]{\ensuremath{\begin{matrix}{rrr} #1 \end{matrix}}}
\newcommand{\cross}{\ensuremath{\times}}
\newcommand{\quant}[1]{\left({#1}\right)}
\newcommand{\abs}[1]{|{#1}|}
\newcommand{\pderiv}[2]{\frac{\partial #1}{\partial #2}}

% Figure commands
\def\centerfig#1{\hbox to\columnwidth{\hss#1\hss}}
\def\centertwofig#1#2{\centerfig{\hbox{\epsfig{figure=#1,width=\@dblcapwidth}\hskip\capskip\psfig{figure=#2,width=\@dblcapwidth}}}}
