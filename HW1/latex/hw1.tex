% Some helpful commands for making CS223A handouts
% Note: this file may not be exactly the same as
% versions from previous years.

% The document class options
\documentclass[11pt,oneside,notitlepage]{article}

% Packages
\usepackage{epsfig}
\usepackage{enumerate}
\usepackage{amsmath}
\usepackage{color}
\usepackage{ulem}
\usepackage{enumitem}

% Preamble (style parameters, etc)
\pagestyle{empty}

\setlength{\paperwidth}{8.5in}
\setlength{\paperheight}{11in}
\setlength{\textwidth}{6.5in}
\setlength{\textheight}{9.5in}

\setlength{\topmargin}{0in}
\setlength{\evensidemargin}{0in}
\setlength{\oddsidemargin}{0in}

\setlength{\headheight}{0in}
\setlength{\headsep}{0in}

% Command to display the header
\newcommand{\HandoutHeader}[1]{
\noindent{\Large{\textbf{Introduction to Robotics (CS223A)}} \hfill Handout {#1}} \rule[1.0mm]{6.5in}{1mm} {\Large (Winter 2016/2017)}
\newline\newline }

\newcommand{\HomeworkHeader}[2]{
\noindent{\newline\newline\newline\Large{\textbf{Introduction to Robotics (CS223A)}} \hfill Homework \#{#1}} \rule[1.0mm]{6.5in}{1mm}
\\
{\Large (Winter 2017/2018)} \hfill {\large Due: \textbf{#2}}\newline\newline }

\newcommand{\HomeworkSolutionsHeader}[2]{
\noindent{\newline\newline\newline\Large{\textbf{Introduction to Robotics (CS223A)}} \hfill Homework \#{#1} Solution} \rule[1.0mm]{6.5in}{1mm} {\Large (Winter 2017/2018)}\newline\newline }


%\newcommand{\HomeworkSolutionsHeader}[1]{
%{\Large{\bfseries Homework \#{#1} solutions}}
%}

\newcommand{\MidtermHeader}[1]{
\noindent{\Large{\bfseries Introduction to Robotics (CS223A)}}\newline
\rule[1.0mm]{6.5in}{1mm}
{\Large (Winter 2017/2018)} \newline\newline\newline
{\Large{\bfseries Midterm}} \hfill {\large Date: \textbf{#1}}\newline\newline\newline
}

\newcommand{\MidtermSolutionsHeader}[1]{
\newline
{\Large{\bfseries Midterm Solutions}} \hfill {\large Date: \textbf{#1}}\newline\newline\newline
}

\newcommand{\FinalHeader}[1]{
\noindent{\Large{\bfseries Introduction to Robotics (CS223A)}}\newline
\rule[1.0mm]{6.5in}{1mm}
{\Large (Winter 2017/2018)} \newline\newline\newline
{\Large{\bfseries Final}} \hfill {\large Date: \textbf{#1}}\newline\newline\newline
}

\newcommand{\FinalSolutionsHeader}[1]{
\newline
{\Large{\bfseries Final Solutions}} \hfill {\large Date: \textbf{#1}}\newline\newline\newline
}

% Helper commands

\newcommand{\q}{\ensuremath{\V{}{q}}}
\newcommand{\qdot}{\ensuremath{\V{}{\dot{q}}}}
\newcommand{\qddot}{\ensuremath{\V{}{\ddot{q}}}}
%\newcommand{\p}[2]{\ensuremath{{^{#1}{\bf P}}
\newcommand{\F}[1]{\ensuremath{\{#1\}}}
\newcommand{\R}[2]{\ensuremath{^{#1}_{#2}R}}
\newcommand{\T}[2]{\ensuremath{^{#1}_{#2}T}}
\newcommand{\U}[2]{\ensuremath{\hat{#1}_{#2}}}
\newcommand{\V}[2]{\ensuremath{^{#1}{\bf #2}}}
\newcommand{\degs}{\ensuremath{^\circ}}
\newcommand{\inv}{\ensuremath{^{-1}}}
\renewenvironment{matrix}{\left[\begin{array}}{\end{array}\right]}
\newenvironment{Tmatrix}{\begin{matrix}{rrrr}}{\end{matrix}}
\newcommand{\Col}[1]{\ensuremath{\begin{matrix}{r} #1 \end{matrix}}}
\newcommand{\Row}[1]{\ensuremath{\begin{matrix}{rrr}#1 \end{matrix}}}
\newcommand{\mat}[1]{\ensuremath{\begin{matrix}{rrr} #1 \end{matrix}}}
\newcommand{\cross}{\ensuremath{\times}}
\newcommand{\quant}[1]{\left({#1}\right)}
\newcommand{\abs}[1]{|{#1}|}
\newcommand{\pderiv}[2]{\frac{\partial #1}{\partial #2}}

% Figure commands
\def\centerfig#1{\hbox to\columnwidth{\hss#1\hss}}
\def\centertwofig#1#2{\centerfig{\hbox{\epsfig{figure=#1,width=\@dblcapwidth}\hskip\capskip\psfig{figure=#2,width=\@dblcapwidth}}}}


\begin{document}

\HomeworkHeader{1}{Wednesday, January 24, 4:00pm}


{\bf Name \underline{\hspace{5cm}} }  \hspace{2cm}   {\bf SUNet ID \underline{\hspace{5cm}} }
\vspace{0cm}


%%%%%%%%%%%%%%%%%%%%%%%%%%%%%%%%%%%%%%%%%%%%%%%%%%%%%%%%%%
\small

Instructions:
\begin{itemize}

\item {\bf Print this problem set and fill in your answers in the dedicated white space right below the question statement. Alternatively you may also submit a typewritten writeup. The assignments should be submitted via Gradescope.}

\item Setup instructions for programming portions are detailed in the README file included with the code.


\item Use abbreviations for trigonometric functions (e.g. $c\theta$ for $\cos(\theta)$,  
$s_1$ or $s\theta_1$ for $\sin(\theta_1)$) in situations where it would be tedious 
to repeatedly write $\sin$,$\cos$, etc.

\item Unless instructed otherwise, leave square roots in symbolic
form rather than writing out their decimal values.

\item If you give a vector as an answer, make sure that you specify
what frame it is given in (if it is not clear from context).  The same
rule applies to rotation and transformation matrices.

\end{itemize}
\vspace{1ex}

\normalsize

%%%%%%%%%%%%%%%%%%%%%%%%%%%%%%%%%%%%%%%%%%%%%%%%%%%%%%%%%%
\begin{enumerate}

%%%%%% Question 1
\item 
The vector $\V{A}{P}$ is first rotated about $\U{X}{A}$ by $\theta$ degrees and then subsequently rotated about $\hat{Y}_{A}$ by $\phi$ degrees.

\begin{enumerate}
  \item[(a)]
  Give the $3 \times 3$ rotation matrix that accomplishes these rotations in the given order. You may leave your answer as a product of matrices. 

\vspace{5cm}


  \item[(b)]
  What is the rotated vector if $\V{A}{P} = [2,1,1]^T$, $\theta = 60^\circ$, and $\phi = 45^\circ$? Be sure to specify the frame of your vector representation.
  \begin{enumerate}
  \item 
  Implement the appropriate functions in  \textit{rotations.py} (where it says \textit{HW1 Q1b: Rotation operators}). This part of the homework will be graded with Gradescope Autograder.
  \item Use the above functions to compute the answer and give your answer on your solution paper.
  \end{enumerate}


\vspace{6cm}


\end{enumerate}
\vspace{1ex}

%%%%%% Question 2
\item
A frame \{B\} is initially coincident with a frame \{A\}. First, we rotate \{B\} about $\hat{Z}_{B}$ by $\theta$ degrees. Next, we rotate the resulting frame \{B\} about $\hat{Y}_{B}$ by $\phi$. Finally, we rotate the resulting frame \{B\} about $\hat{X}_{B}$ by $\theta$ again.
\begin{enumerate}[label=(\alph*)]
  \item
   Determine the $3 \times 3$ rotation matrix, $\R{A}{B}$, that will change the description of a vector $P$ in frame \{B\}, $\V{B}{P}$, to frame \{A\}, $\V{A}{P}$. You may leave your answer as a product of matrices. 
   
  
\vspace{6cm}
  \item
What is the value of $\R{A}{B}$ if $\theta = 45^\circ$, $\phi = 60^\circ$? Implement the appropriate functions in \textit{rotations.py} (where it says \textit{HW1 Q2b: Euler and Fixed angles}) and use them to compute the answer. Handle representation singularities explicitly in your code. Similar to Q1, please give your answer on your solution paper.

\vspace{3cm}
  \item
Compute the rotation matrix if $\phi = 90^\circ$? Can you achieve the same final \{B\} using a different set of Euler Angles? Explain why.
\vspace{4cm}
  \end{enumerate}
\vspace{1ex}
  
\newpage
  
%%%%%% Question 3
\item
Frame \{A\} and frame \{B\} are fixed with respect to an inertial ground frame.

\begin{enumerate}[label=(\alph*)]
  \item Consider a velocity vector in frame \{A\}, $\V{A}{V}$. How will it change if
we express it in frame \{B\}? Are $\V{A}{V}$ and $\V{B}{V}$ the same vector? Comment upon their magnitudes and directions. If they are different how can you transform one into the other?


  \item
Given 
\begin{eqnarray*}
\V{A}{V} = 
  \begin{Tmatrix}
    -4 \\
    z \\
    -6 \\
  \end{Tmatrix}, 
  \V{B}{V} = 
  \begin{Tmatrix}
   -6 \\
    4 \\
    1 \\
  \end{Tmatrix},
\end{eqnarray*}
where $\V{A}{V}$, $\V{B}{V}$ have the same meaning as in part (a). Assume that transforming frame \{A\} into frame \{B\} requires translating \{A\} by $(0,2,1)$ and then rotating it $90^\circ$ about $\U{X}{A^\prime}$, and rotating the resulting frame $\theta$ about $\U{Z}{A^{\prime\prime}}$ (Note that $A^{\prime}$ and $A^{\prime\prime}$ are intermediate frames). Determine the value of $\theta$ and $z$.




\end{enumerate}
\vspace{1ex}

\newpage

%%%%%% Question 4
\item
Given the following transformation matrices:
  \begin{eqnarray*}
    T1 = \begin{Tmatrix}
        \frac{\sqrt{3}}{2} &  -\frac{1}{2} & 0 & 1\\
        -\frac{1}{2} & \frac{\sqrt{3}}{2} & 0 & 0\\
        0 & 0 & 1 & 0\\
        0 & 0 & 0 & 1
    \end{Tmatrix},
    T2 = \begin{Tmatrix}
        \frac{\sqrt{3}}{2} & 0 & -\frac{1}{2}  & -1\\
        0 & 1 & 0 & 2\\
        \frac{1}{2} &  0 &\frac{\sqrt{3}}{2} & 0\\
        0 & 0 & 1 & 0
    \end{Tmatrix}, 
    \\
    T3 = \begin{Tmatrix}
        \frac{1}{2} &  \frac{\sqrt{3}}{2} & 0 & 1\\
        -\frac{\sqrt{3}}{2} & \frac{1}{2} & 0 & 2\\
        0 & 0 & 1 & 0\\
        0 & 0 & 0 & 1
    \end{Tmatrix},
    T4 = \begin{Tmatrix}
        \frac{1}{2} &  \frac{\sqrt{3}}{2} & 0 & 0\\
         \frac{\sqrt{3}}{2} & -\frac{1}{2} & 0 & 0\\
        0 & 0 & 1 & 0\\
        0 & 0 & 0 & 1
    \end{Tmatrix}.
  \end{eqnarray*}

    

  
  

\begin{enumerate}[label=(\alph*)]
    \item
    Are T1, T2, T3 and T4 valid transformation matrices? Explain why or why not, and if there are multiple reasons why a matrix is invalid, include each. (We define a transformation matrix as a rotation and a translation, i.e. a "rigid body" transformation)
    
    \vspace{5cm}
    
    \item
    Implement function \textit{mat\_to\_quat} in \textit{rotations.py} and find the Euler parameters that represent the rotations for the correct matrix (or matrices). 
    
    \vspace{5cm}
    
    \item
   Also find the unit vector that defines the axis of rotation, and the angle of rotation for the correct matrix (or matrices). 
\end{enumerate}
\newpage

%%%%%% Question 5
\item
  \begin{enumerate}
  \item[(a)] Prove the following Lemma :
  
  \textit{For all rotations, at least one of the Euler parameters is greater than or equal to 1/2.}
  
  \vspace{5cm}

  \item[(b)] When possible, determine the Euler parameters $\varepsilon_1, \varepsilon_2, \varepsilon_3, \varepsilon_4$ for the following
    matrices:

    \[
    R_1 =
    \left[
      \begin{array}{rrr}
	1 & 0 & 0 \\
    0 & 1 & 0 \\
    0 & 0 & 1
      \end{array}
    \right]
    \hspace{0.5cm}
    R_2 =
    \left[
      \begin{array}{rrr}
	0 & -1 & 0 \\
    1 & 0 & 0 \\
    0 & 0 & 1
      \end{array}
    \right]
    \hspace{0.5cm}
    R_3 =
    \left[
      \begin{array}{rrr}
	1 & 1 & 1 \\
    1 & 1 & 1 \\
    1 & 1 & 1
      \end{array}
    \right]
    \hspace{0.5cm}
    R_4 =
    \left[
      \begin{array}{rrr}
	-\frac{1}{2} &  \frac{1}{2} & \frac{1}{\sqrt{2}} \\
        \frac{1}{2} & -\frac{1}{2} & \frac{1}{\sqrt{2}} \\
        \frac{1}{\sqrt{2}} & \frac{1}{\sqrt{2}} & 0
      \end{array}
    \right]
    \]

  \end{enumerate}
\newpage

%%%%%% Question 6
\item 
\begin{enumerate}
\item[(a)] Prove that 2D rotations commute.

\vspace{9cm}

\item[(b)] Prove that 3D rotations do not necessarily commute.

\newpage

\item[(c)] \textbf{[Extra Credit]} Prove that 3D infinitesimal rotations commute (clearly state the assumptions you make and their consequences).
\end{enumerate}


\end{enumerate}

\end{document}

